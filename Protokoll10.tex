% Für Bindekorrektur als optionales Argument "BCORfaktormitmaßeinheit", dann
% sieht auch Option "twoside" vernünftig aus
% Näheres zu "scrartcl" bzw. "scrreprt" und "scrbook" siehe KOMA-Skript Doku
\documentclass[12pt,a4paper,titlepage,headinclude,bibtotoc]{scrartcl}


%---- Allgemeine Layout Einstellungen ------------------------------------------

% Für Kopf und Fußzeilen, siehe auch KOMA-Skript Doku
\usepackage[komastyle]{scrpage2}
\pagestyle{scrheadings}
\automark[section]{chapter}
\setheadsepline{0.5pt}[\color{black}]

%keine Einrückung
\parindent0pt

%Einstellungen für Figuren- und Tabellenbeschriftungen
\setkomafont{captionlabel}{\sffamily\bfseries}
\setcapindent{0em}

\usepackage{caption}

%---- Weitere Pakete -----------------------------------------------------------
% Die Pakete sind alle in der TeX Live Distribution enthalten. Wichtige Adressen
% www.ctan.org, www.dante.de

% Sprachunterstützung
\usepackage[ngerman]{babel}

% Benutzung von Umlauten direkt im Text
% entweder "latin1" oder "utf8"
\usepackage[utf8]{inputenc}

% Pakete mit Mathesymbolen und zur Beseitigung von Schwächen der Mathe-Umgebung
\usepackage{latexsym,exscale,amssymb,amsmath}

% Weitere Symbole
\usepackage[nointegrals]{wasysym}
\usepackage{eurosym}

% Anderes Literaturverzeichnisformat
%\usepackage[square,sort&compress]{natbib}

% Für Farbe
\usepackage{color}

% Zur Graphikausgabe
%Beipiel: \includegraphics[width=\textwidth]{grafik.png}
\usepackage{graphicx}

% Text umfließt Graphiken und Tabellen
% Beispiel:
% \begin{wrapfigure}[Zeilenanzahl]{"l" oder "r"}{breite}
%   \centering
%   \includegraphics[width=...]{grafik}
%   \caption{Beschriftung} 
%   \label{fig:grafik}
% \end{wrapfigure}
\usepackage{wrapfig}

% Mehrere Abbildungen nebeneinander
% Beispiel:
% \begin{figure}[htb]
%   \centering
%   \subfigure[Beschriftung 1\label{fig:label1}]
%   {\includegraphics[width=0.49\textwidth]{grafik1}}
%   \hfill
%   \subfigure[Beschriftung 2\label{fig:label2}]
%   {\includegraphics[width=0.49\textwidth]{grafik2}}
%   \caption{Beschriftung allgemein}
%   \label{fig:label-gesamt}
% \end{figure}
\usepackage{subfigure}
\usepackage{adjustbox}

% Caption neben Abbildung
% Beispiel:
% \sidecaptionvpos{figure}{"c" oder "t" oder "b"}
% \begin{SCfigure}[rel. Breite (normalerweise = 1)][hbt]
%   \centering
%   \includegraphics[width=0.5\textwidth]{grafik.png}
%   \caption{Beschreibung}
%   \label{fig:}
% \end{SCfigure}
\usepackage{sidecap}

% Befehl für "Entspricht"-Zeichen
\newcommand{\corresponds}{\ensuremath{\mathrel{\widehat{=}}}}

%Für chemische Formeln (von www.dante.de)
%% Anpassung an LaTeX(2e) von Bernd Raichle
\makeatletter
\DeclareRobustCommand{\chemical}[1]{%
  {\(\m@th
   \edef\resetfontdimens{\noexpand\)%
       \fontdimen16\textfont2=\the\fontdimen16\textfont2
       \fontdimen17\textfont2=\the\fontdimen17\textfont2\relax}%
   \fontdimen16\textfont2=2.7pt \fontdimen17\textfont2=2.7pt
   \mathrm{#1}%
   \resetfontdimens}}
\makeatother

%Si Einheiten
\usepackage{siunitx}

%c++ Code einbinden
\usepackage{listings}
\lstset{numbers=left, numberstyle=\tiny, numbersep=5pt}

%Differential
\newcommand{\dif}{\ensuremath{\mathrm{d}}}

%Boxen,etc.
\usepackage{fancybox}
\usepackage{empheq}

%Fußnoten auf gleiche Seite
\interfootnotelinepenalty=1000

%Dateien aus Unterverzeichnissen
\usepackage{import}

%Bibliography \bibliography{literatur} und \cite{gerthsen}
%\usepackage{cite}
\usepackage{babelbib}
\selectbiblanguage{ngerman}

\begin{document}

\begin{titlepage}
\centering
\textsc{\Large Anfängerpraktikum der Fakultät für
  Physik,\\[1.5ex] Universität Göttingen}

\vspace*{4.2cm}

\rule{\textwidth}{1pt}\\[0.5cm]
{\huge \bfseries
  Die Potenzialwaage\\[1.5ex]
  Protokoll:}\\[0.5cm]
\rule{\textwidth}{1pt}

\vspace*{3.0cm}

\begin{Large}
\begin{tabular}{ll}
Praktikant:
 	&  Felix Kurtz\\
 	&  Michael Lohmann\\

  E-Mail: 
	&  felix.kurtz@stud.uni-goettingen.de\\
	& m.lohmann@stud.uni-goettingen.de\\

 Betreuer: & Björn Klaas\\
 Versuchsdatum: & 04.09.2014\\
\end{tabular}
\end{Large}

\vspace*{0.8cm}

\begin{Large}
\fbox{
  \begin{minipage}[t][2.5cm][t]{6cm} 
    Testat:
  \end{minipage}
}
\end{Large}

\end{titlepage}

\tableofcontents

\newpage

\section{Einleitung}
\label{sec:einleitung}
In diesem Versuch soll die \textit{elektrische Feldkonstante} gemessen werden.
Sie ist eine der wichtigsten Größen in der Elektrodynamik.
Zur Messung verwenden wir die \textit{Kirchhoffsche Potentialwaage}.
Hierbei werden die Kräfte ausgenutzt, die zwischen den Platten eines Kondensators wirken.

\section{Theorie}
\label{sec:theorie}
Die Kapazität $C$ eines Plattenkondenstaors mit dem Plattenabstand $d$ und der Plattenfläche $A$ berechnet sich nach der folgenden Formel:
\begin{align}
 C=\varepsilon_r\varepsilon_0\frac{A}{d}
 \label{eq:C_Pl}
\end{align}
Dabei ist $\varepsilon_0$ die elektrische Feldkonstante und $\varepsilon_r$ die Permitivität des Mediums, welches sich zwischen den Platten befindet.
Da unser Versuch in Luft stattfindet, wird im folgenden mit $\varepsilon_r=1$ gerechnet.\\
Um die Energie, die in einem Kondensator gespeichert ist, zu bestimmen, integriert man die Ladung $Q$ nach $\dif U$.
Da $Q=C\cdot U$ gilt, erhält man:
\begin{align}
 W=\frac{1}{2} C U^2
\end{align}
Da die Kraft $F$, die zwischen den beiden Platten des Kondensators wirkt, der Gradient der Energie ist, folgt mit \eqref{eq:C_Pl}:
%Zwischenschritt
\begin{align}
 F=-\frac{\dif W}{\dif d}= -\frac{1}{2}\varepsilon_0\frac{\dif}{\dif d}\frac{A}{d} \;U^2 =\varepsilon_0\frac{A U^2}{2 d^2}
 \label{eq:F_Pl}
\end{align}
Im Gleichgewichtsfall der Kichhoffschen Potentialwaage wird diese Kraft mit der Gewichtskraft $F_G$ des Wägstückes gleichgesetzt (siehe Abschnitt \ref{sec:durchfuehrung}):
\begin{align}
 \varepsilon_0\frac{A U^2}{2 d^2}=mg
 \label{eq:PotWaage}
\end{align}

\section{Durchführung}
\label{sec:durchfuehrung}
In Abbildung \ref{fig:PotWaageSchema} ist die Potentialwaage schematisch dargestellt.
Bei diesem Versuch sollten folgende \textbf{Hinweise} beachtet werden:
Die Gewichte sollten nur mit einer Pinzette auf die Waage gelegt werden, da schon der kleinste Fettfleck ihre Masse beträchtlich ändern könnte und so die Messung verfälscht.
Außerdem muss bei Betrieb der Waage, also angelegter Hochspannung, aus Sicherheitsgründen das Fenster geschlossen sein.\\
\begin{figure}[!htb]
	\centering	
	\includegraphics[scale=0.7]{PotWaageSchema.png}
	\caption{Schema der Potentialwaage: \small{\emph{1. Gewichtauflage; 2. obere Kondensatorplatte; 3. untere Kondensatorplatte; 4. Lot; 5. Arretierung; 6. Justierrad der unteren Kondensatorplatte; 7. Justierschrauben; 8. Holz-/Glas-Gehäuse}}\protect\footnotemark}
	\label{fig:PotWaageSchema}
\end{figure}
\footnotetext{Quelle: https://lp.uni-goettingen.de/get/text/3664, abgerufen am 06.09.2014}



\textbf{Messung 1:}
Zuerst wird die Waage arretiert, um 3g auf die Waagschale zu legen.
Dann wird eine Spannung $U$ (2, 3, 4, 5 kV) angelegt und die Arretierung gelöst.
Sollte die Waage kippen, muss der Plattenabstand $d$ des Kondensators verkleinert werden.
Anschließend wird der Plattenabstand vergrößert, bis die obere Kondensatorplatte abgehoben wird.
Diesen Wert notiert man dann.
All dies wird noch für eine aufgelegte Masse von $m=4$g wiederholt.\\  
\textbf{Messung 2:}
Nun wird der Plattenabstand auf $d=2, 2.5, 3, 4\,$mm fest eingestellt.
Für mindestens 3 verschiedene Massen $1\,\text{g}\leq m \leq 4\,\text{g}$ wird durch Vermindern der angelegten Spannung $U$ die Spannung bestimmt, bei der die Waage kippt.

\section{Auswertung}
\label{sec:auswertung}
Bevor mit der eigentlichen Auswertung begonnen wird, berechnen wir die effektive Fläche $A$ des Kondensators, da hier die kapazitiven Effekte zwischen Ring und Platte beachtet werden müssen.
Diese berechnet man nach der Formel aus dem Praktikumshandbuch:
\begin{align*}
 A=\pi (r^2+ra)
\end{align*}
Dabei ist $r=40\,$mm der Radius der oberen Platte ohne Schutzring und $a=1\,$mm die Breite des Schlitzes.
So ergibt sich: $$A=5.152 \cdot 10^{-3}\,\si{\meter^2}$$
Im Folgenden wird mit einer Erdbeschleunigung von $g=9.81\,\si{\meter\per\second^2}$ gerechnet.

\subsection{konstante Kraft}
Aus Gleichung \eqref{eq:PotWaage} folgt eine lineare Abhängigkeit zwischen der Spannung $U$ und dem Plattenabstand $d$ für eine konstante Gewichtskraft, also wenn die Masse $m$ fest ist.
\begin{align}
 d= \sqrt{\frac{\varepsilon_0 A}{2mg}} \cdot U
\end{align}
In der Abbildung \ref{fig:d(U)} ist diese Abhängigkeit dargestellt.
Aus der Geradensteigung $k$ lässt sich durch Umstellen der obigen Formel $\varepsilon_0$ berechnen:
\begin{align*}
 \varepsilon_0=\frac{2mg}{A}\cdot k^2
\end{align*}

\begin{figure}[!htb]
 \centering
 \input{messung1.tex}
 \caption{Plattenabstand in Abängigkeit der angelegten Spannung}
 \label{fig:d(U)}
\end{figure}
Es ergeben sich diese Steigungen $k$ und die daraus resultierenden Werte für $\varepsilon_0$:
\begin{table}[!htb]
 \centering
 \begin{tabular}{|c|c|c|}
  \hline
  \rule{0pt}{15pt}$m~[\si{\gram}]$&  $k~\left[\si[per-mode=fraction]{\milli\meter\per\kilo\volt}\right]$ & $\varepsilon_0~\left[10^{-12}\,\si[per-mode=fraction]{\ampere\second\per\volt\per\meter}\right]$\\
  \hline
  $3$ & $0.739\pm 0.017$ & $6.24 \pm 0.29$\\
  $4$ & $0.650\pm 0.007$ & $6.44 \pm 0.14$\\
  \hline
 \end{tabular}
 \caption{Geradensteigung und daraus berechnete elektrische Feldkonstante}
 \label{tab:messung1_k_epsilon0}
\end{table}

Es ergibt sich ein gewichteter Mittelwert von
\begin{empheq}[box=\shadowbox*]{align*}
  \overline{\varepsilon_0}=(6.40 \pm 0.13)\cdot 10^{-12}\,\si[per-mode=fraction]{\ampere\second\per\volt\per\meter}
\end{empheq}

Aus der linearen Regression folgt der Offset $\Delta$ des Plattenabstandes als das Negative des y-Achsenabschnittes:\\
\begin{table}[!htb]
 \centering
 \begin{tabular}{|c|c|}
  \hline
  \rule{0pt}{15pt}$m~[\si{\gram}]$&  $\Delta ~ [\si{\milli\meter}]$\\
  \hline
  $3$ & $0.56 \pm 0.06$\\
  $4$ & $0.650\pm 0.007$\\
  \hline
 \end{tabular}
 \caption{Offset des Abstandes}
 \label{tab:messung1_Delta}
\end{table}

Es ergibt sich ein gewichteter Mittelwert von
\begin{empheq}[box=\shadowbox*]{align*}
  \overline{\Delta}=(0.595 \pm 0.022)\,\si{\milli\meter}
\end{empheq}
Für den wahren Plattenabstand $d_w$ muss also zum gemessenen Wert $d$ noch $\Delta$ addiert werden.
Da wir von einer Unsicherheit $\sigma_d=0.01\,$mm ausgehen, ergibt sich für den Fehler des wahren Abstandes $$\sigma_{d_w}=\sqrt{\sigma_d^2+\sigma_\Delta^2}=0.024\,\si{\milli\meter}\;.$$

\subsection{konstanter Plattenabstand}
Lässt man nun aber den Plattenabstand $d$ konstant ergibt sich direkt aus \eqref{eq:PotWaage} eine proportionale Abhängigkeit zwischen der Kraft $F_G$ und dem Quadrat der Spannung $U$.
Diese ist in Abb.\ref{fig:F(U^2)} für die 4 verschiedenen Plattenabstände $d$, die wir vermessen haben, zu erkennen.

\begin{figure}[!htb]
 \centering
 \input{messung2.tex}
 \caption{Kraft in Abängigkeit des Quadrats der angelegten Spannung}
 \label{fig:F(U^2)}
\end{figure}

Aus der Geradensteigung $k$ lässt sich wieder $\varepsilon_0$ bestimmen.
Dabei muss jedoch der wahre Plattenabstand aus dem obigen Abschnitt verwendet werden.
Des Weiteren folgt aus der Fehlerfortpflanzung die untere der beiden Formeln.
\begin{align}
 \varepsilon_0 &= k\cdot \frac{2d_w^2}{A}\\
 \sigma_{\varepsilon_0} &= \sqrt{\sigma_k^2\cdot\left(\frac{2d_w^2}{A}\right)^2+\sigma_{d_w}^2\cdot\left(\frac{4d_wk}{A}\right)^2}
\end{align}

In Tabelle \ref{tab:messung2} sind neben dem Plattenabstand $d$ und $d_w$ die Geradensteigung $k$ und der resultierende Wert für $\varepsilon_0$ aus unseren Messreihen zu finden.

\begin{table}[!htb]
	\centering
	\begin{tabular}{|c|c|c|c|}
		\hline
		\rule{0pt}{15pt}$d~[\si{\milli\meter}]$ &  $d_w ~ [\si{\milli\meter}]$ & $k$ [N/(kV)$^2$] & $\varepsilon_0 ~ \left[10^{-12}\,\si[per-mode=fraction]{\ampere\second\per\volt\per\meter}\right]$\\
		\hline
		$2.000 \pm 0.010$ & $2.595 \pm 0.024$ & $0.00226 \pm 0.00005$ & $5.92 \pm 0.16$ \\
$2.500 \pm 0.010$ & $3.095 \pm 0.024$ & $0.001680 \pm 0.000026$ & $6.25 \pm 0.14$ \\
$3.000 \pm 0.010$ & $3.595 \pm 0.024$ & $0.001288 \pm 0.000022$ & $6.46 \pm 0.14$ \\
$4.000 \pm 0.010$ & $4.595 \pm 0.024$ & $0.000870 \pm 0.000022$ & $7.13 \pm 0.20$ \\
		\hline
	\end{tabular}
	\caption{Werte dieser Messung}
	\label{tab:messung2}
\end{table}

Für $\varepsilon_0$ ergibt sich ein gewichteter Mittelwert von
\begin{empheq}[box=\shadowbox*]{align*}
  \overline{\varepsilon_0}=(6.37 \pm 0.08)\cdot 10^{-12}\,\si[per-mode=fraction]{\ampere\second\per\volt\per\meter}
\end{empheq}


\section{Diskussion}
\label{sec:diskussion}
Vergleicht man beide Werte für $\varepsilon_0$ untereinander, fällt auf, dass sie weniger als $1\%$ voneinander abweichen und der eine Wert im Fehlerintervall des jeweils anderen liegt.
Verglichen mit dem Literaturwert $\varepsilon_0\approx 8.854\cdot 10^{-12}\,\si[per-mode=fraction]{\ampere\second\per\volt\per\meter}$ ergibt sich jedoch eine erhebliche Abweichung von mehr als $25\%$.
Dies deutet also auf einen systematischen Fehler im Versuchsaufbau hin.


Die Annahme $\varepsilon_r=1$ kann auch weiterhin verwendet werden, da für alle anderen Medien als Vakuum $\varepsilon_r$ größer ist und so der gemessene Wert für $\varepsilon_0$ größer sein müsste.\\

Außerdem stellt man fest, dass die $\varepsilon_0$-Werte aus Tabelle \ref{tab:messung2} nicht nur sehr streuen, sondern auch mit zunehmendem Plattenabstand größer werden.
Wir haben allerdings keine Erklärung für diesen merkwürdigen Zusammenhang.

\section{Anhang}

\end{document}
