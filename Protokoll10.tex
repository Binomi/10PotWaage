% Für Bindekorrektur als optionales Argument "BCORfaktormitmaßeinheit", dann
% sieht auch Option "twoside" vernünftig aus
% Näheres zu "scrartcl" bzw. "scrreprt" und "scrbook" siehe KOMA-Skript Doku
\documentclass[12pt,a4paper,titlepage,headinclude,bibtotoc]{scrartcl}


%---- Allgemeine Layout Einstellungen ------------------------------------------

% Für Kopf und Fußzeilen, siehe auch KOMA-Skript Doku
\usepackage[komastyle]{scrpage2}
\pagestyle{scrheadings}
\automark[section]{chapter}
\setheadsepline{0.5pt}[\color{black}]

%keine Einrückung
\parindent0pt

%Einstellungen für Figuren- und Tabellenbeschriftungen
\setkomafont{captionlabel}{\sffamily\bfseries}
\setcapindent{0em}

\usepackage{caption}

%---- Weitere Pakete -----------------------------------------------------------
% Die Pakete sind alle in der TeX Live Distribution enthalten. Wichtige Adressen
% www.ctan.org, www.dante.de

% Sprachunterstützung
\usepackage[ngerman]{babel}

% Benutzung von Umlauten direkt im Text
% entweder "latin1" oder "utf8"
\usepackage[utf8]{inputenc}

% Pakete mit Mathesymbolen und zur Beseitigung von Schwächen der Mathe-Umgebung
\usepackage{latexsym,exscale,amssymb,amsmath}

% Weitere Symbole
\usepackage[nointegrals]{wasysym}
\usepackage{eurosym}

% Anderes Literaturverzeichnisformat
%\usepackage[square,sort&compress]{natbib}

% Für Farbe
\usepackage{color}

% Zur Graphikausgabe
%Beipiel: \includegraphics[width=\textwidth]{grafik.png}
\usepackage{graphicx}

% Text umfließt Graphiken und Tabellen
% Beispiel:
% \begin{wrapfigure}[Zeilenanzahl]{"l" oder "r"}{breite}
%   \centering
%   \includegraphics[width=...]{grafik}
%   \caption{Beschriftung} 
%   \label{fig:grafik}
% \end{wrapfigure}
\usepackage{wrapfig}

% Mehrere Abbildungen nebeneinander
% Beispiel:
% \begin{figure}[htb]
%   \centering
%   \subfigure[Beschriftung 1\label{fig:label1}]
%   {\includegraphics[width=0.49\textwidth]{grafik1}}
%   \hfill
%   \subfigure[Beschriftung 2\label{fig:label2}]
%   {\includegraphics[width=0.49\textwidth]{grafik2}}
%   \caption{Beschriftung allgemein}
%   \label{fig:label-gesamt}
% \end{figure}
\usepackage{subfigure}
\usepackage{adjustbox}

% Caption neben Abbildung
% Beispiel:
% \sidecaptionvpos{figure}{"c" oder "t" oder "b"}
% \begin{SCfigure}[rel. Breite (normalerweise = 1)][hbt]
%   \centering
%   \includegraphics[width=0.5\textwidth]{grafik.png}
%   \caption{Beschreibung}
%   \label{fig:}
% \end{SCfigure}
\usepackage{sidecap}

% Befehl für "Entspricht"-Zeichen
\newcommand{\corresponds}{\ensuremath{\mathrel{\widehat{=}}}}

%Für chemische Formeln (von www.dante.de)
%% Anpassung an LaTeX(2e) von Bernd Raichle
\makeatletter
\DeclareRobustCommand{\chemical}[1]{%
  {\(\m@th
   \edef\resetfontdimens{\noexpand\)%
       \fontdimen16\textfont2=\the\fontdimen16\textfont2
       \fontdimen17\textfont2=\the\fontdimen17\textfont2\relax}%
   \fontdimen16\textfont2=2.7pt \fontdimen17\textfont2=2.7pt
   \mathrm{#1}%
   \resetfontdimens}}
\makeatother

%Si Einheiten
\usepackage{siunitx}

%c++ Code einbinden
\usepackage{listings}
\lstset{numbers=left, numberstyle=\tiny, numbersep=5pt}

%Differential
\newcommand{\dif}{\ensuremath{\mathrm{d}}}

%Boxen,etc.
\usepackage{fancybox}
\usepackage{empheq}

%Fußnoten auf gleiche Seite
\interfootnotelinepenalty=1000

%Dateien aus Unterverzeichnissen
\usepackage{import}

%Bibliography \bibliography{literatur} und \cite{gerthsen}
%\usepackage{cite}
\usepackage{babelbib}
\selectbiblanguage{ngerman}

\begin{document}

\begin{titlepage}
\centering
\textsc{\Large Anfängerpraktikum der Fakultät für
  Physik,\\[1.5ex] Universität Göttingen}

\vspace*{4.2cm}

\rule{\textwidth}{1pt}\\[0.5cm]
{\huge \bfseries
  Die Potenzialwaage\\[1.5ex]
  Protokoll:}\\[0.5cm]
\rule{\textwidth}{1pt}

\vspace*{3.0cm}

\begin{Large}
\begin{tabular}{ll}
Praktikant:
 	&  Felix Kurtz\\
 	&  Michael Lohmann\\

  E-Mail: 
	&  felix.kurtz@stud.uni-goettingen.de\\
	& m.lohmann@stud.uni-goettingen.de\\

 Betreuer: & Björn Klaas\\
 Versuchsdatum: & 04.09.2014\\
\end{tabular}
\end{Large}

\vspace*{0.8cm}

\begin{Large}
\fbox{
  \begin{minipage}[t][2.5cm][t]{6cm} 
    Testat:
  \end{minipage}
}
\end{Large}

\end{titlepage}

\tableofcontents

\newpage

\section{Einleitung}
\label{sec:einleitung}


\section{Theorie}
\label{sec:theorie}
<<<<<<< HEAD
\begin{align}
	C=\varepsilon_r\varepsilon_0\frac{A}{d}
\end{align}
=======
Die Kapazität eines Plattenkondenstaors mit dem Plattenabstand $d$ und der Plattenfläche $A$ berechnet sich nach der folgenden Formel:
\begin{align}
 C=\varepsilon_r\varepsilon_0\frac{A}{d}
 \label{eq:C_Pl}
\end{align}
Dabei ist $\varepsilon_0$ die elektrische Feldkonstante und $\varepsilon_r$ die Permitivität des Mediums, welches sich zwischen den Platten befindet.
Da unser Versuch in Luft stattfindet, wird im folgenden mit $\varepsilon_r=1$ gerechnet.\\
Energie, die in einem Kondensator gespeichert ist:
\begin{align}
 W=\frac{1}{2} C U^2
\end{align}
Kraft, die zwischen den beiden Platten des Kondensators wirkt:
\begin{align}
 F=\varepsilon_0\frac{A U^2}{2 d^2}
 \label{eq:F_Pl}
\end{align}
Bei der Kichhoffschen Potentialwaage wird diese Kraft mit der Gewichtskraft des Wägstückes $F_G$ gleichgesetzt:
\begin{align}
 \varepsilon_0\frac{A U^2}{2 d^2}=mg
 \label{eq:PotWaage}
\end{align}

>>>>>>> f01e1ae64c5c96aeae3ad649c6f4bfd80b513754

\section{Durchführung}
\label{sec:durchfuehrung}

\section{Auswertung}
\label{sec:auswertung}
Bevor mit der eigentlichen Auswertung begonnen wird, berechnen wir die effektive Fläche $A$ des Kondensators, da hier die kapazitiven Effekte zwischen Ring und Platte beachtet werden müssen.
Diese berechnet man nach der Formel aus dem Praktikumshandbuch:
\begin{align*}
 A=\pi (r^2+ra)
\end{align*}
Dabei ist $r=40\,$mm der Radius der oberen Platte ohne Schutzring und $a=1\,$mm die Breite des Schlitzes.
So ergibt sich: $$A=5.152 \cdot 10^{-3}\,\si{\meter^2}$$
Im folgenden wird mit einer Erdbeschleunigung von $g=9.81\,\si{\meter\per\second^2}$ gerechnet.

\subsection{konstante Kraft}
Aus Gleichung \eqref{eq:PotWaage} folgt eine lineare Abhängigkeit zwischen der Spannung $U$ und dem Plattenabstand $d$ für eine konstante Gewichtskraft, also wenn die Masse $m$ fest ist.
\begin{align}
 d= \sqrt{\frac{\varepsilon_0 A}{2mg}} \cdot U
\end{align}
In der nachfolgenden Abbildung \ref{fig:d(U)} ist diese Abhängigkeit dargestellt.
\begin{figure}[!htb]
 \centering
 \input{messung1.tex}
 \caption{Plattenabstand in Abängigkeit der angelegten Spannung}
 \label{fig:d(U)}
\end{figure}
Aus der Geradensteigung $k$ lässt sich durch Umstellen der obigen Formel $\varepsilon_0$ berechnen:
\begin{align*}
 \varepsilon_0=\frac{2mg}{A}\cdot k^2
\end{align*}

Es ergeben sich diese Steigungen $k$ und die daraus resultierenden Werte für $\varepsilon_0$:
\begin{table}[!htb]
 \centering
 \begin{tabular}{|c|c|c|}
  \hline
  $m~[\si{\gram}]$&  $k~\left[\si[per-mode=fraction]{\milli\meter\per\kilo\volt}\right]$ & $\varepsilon_0~\left[10^{-12}\,\si[per-mode=fraction]{\ampere\second\per\volt\per\meter}\right]$\\
  \hline
  $3$ & $0.739\pm 0.017$ & $6.24 \pm 0.29$\\
  $4$ & $0.650\pm 0.007$ & $6.44 \pm 0.14$\\
  \hline
 \end{tabular}
 \caption{Geradensteigung und daraus berechnete elektrische Feldkonstante}
 \label{tab:messung1_k_epsilon0}
\end{table}

Es ergibt sich ein gewichteter Mittelwert von
\begin{empheq}[box=\shadowbox*]{align*}
  \overline{\varepsilon_0}=(6.40 \pm 0.13)\cdot 10^{-12}\,\si[per-mode=fraction]{\ampere\second\per\volt\per\meter}
\end{empheq}

Aus der linearen Regression folgt der Offset des Plattenabstandes als y-Achsenabschnitt:\\
\begin{table}[!htb]
 \centering
 \begin{tabular}{|c|c|}
  \hline
  $m~[\si{\gram}]$&  $\Delta ~ [\si{\milli\meter}]$\\
  \hline
  $3$ & $-0.56 \pm 0.06$\\
  $4$ & $0.650\pm 0.007$\\
  \hline
 \end{tabular}
 \caption{Offset des Abstandes}
 \label{tab:messung1_Delta}
\end{table}

Es ergibt sich ein gewichteter Mittelwert von
\begin{empheq}[box=\shadowbox*]{align*}
  \overline{\Delta}=(-0.595 \pm 0.022)\,\si{\milli\meter}
\end{empheq}
Für den wahren Plattenabstand $d_w$ muss also zum gemessenen Wert $d$ noch $\Delta$ addiert werden.

\subsection{konstanter Plattenabstand}
\begin{figure}[!htb]
 \centering
 \input{messung2.tex}
 \caption{Kraft in Abängigkeit des Quadrats der angelegten Spannung}
 \label{fig:F(U^2)}
\end{figure}

\begin{align}
 \varepsilon_0 &= k\cdot \frac{2d^2}{A}\\
 \sigma_{\varepsilon_0} &= \sqrt{\sigma_k^2\cdot\left(\frac{2d^2}{A}\right)^2+\sigma_d^2\cdot\left(\frac{4dk}{A}\right)^2}
\end{align}


\begin{table}[!htb]
 \centering
 \begin{tabular}{|c|c|c|c|}
  \hline
  $d~[\si{\milli\meter}]$ &  $d_w ~ [\si{\milli\meter}]$ & $k$ [N/kV] & $\varepsilon_0 ~ \left[10^{-12}\,\si[per-mode=fraction]{\ampere\second\per\volt\per\meter}\right]$\\
  \hline
  \hline
 \end{tabular}
 \caption{...}
 \label{tab:messung2}
\end{table}


\section{Diskussion}
\label{sec:diskussion}

\section{Anhang}

\end{document}
